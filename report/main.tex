\documentclass[a4paper, twocolumn, 11pt, twoside]{article}
\usepackage[english]{babel} %language
\usepackage[utf8]{inputenc} %input encoding
\usepackage{float} %position of floating objects
\usepackage{bookmark} %hyperlinks in pdf
\usepackage{subcaption}
\usepackage[T1]{fontenc}
\usepackage{lipsum} %placeholder text
\usepackage{amsthm} %theorems
\usepackage{amsmath}
\usepackage{mathtools}
% \usepackage{physics}
% \usepackage{xcolor}
% \usepackage{graphicx}
% \usepackage[left=23mm,right=13mm,top=35mm,columnsep=15pt]{geometry} 
% \usepackage{adjustbox}
% \usepackage{placeins}
% \usepackage{csquotes}
% \usepackage{algorithm} 
% \usepackage{algpseudocode}
% \usepackage{listings}
% \usepackage{graphicx}
% \usepackage{tikz}
% \usepackage[normalem]{ulem}
% \useunder{\uline}{\ul}{}

\title{Long-Horizon Vehicle Motion Planning and Control Through Serially Cascaded Model Complexity}
\author{}
\date{\today}

\begin{document}

\maketitle

\begin{abstract}
    We propose the implementation and experimentation of a motion planning and
    control framework for autonomous vehicles based on nonlinear
    model-predictive control.The work is mainly based on \cite{paper}. The code is available publicly at
    \href{https://github.com/neverorfrog/vehicle-control}{this GitHub
    repository}. 
\end{abstract}


\section*{Introduction}

% Using the star (*) form of the sectioning command suppresses the numbering for
% that particular section or subsection.
Model predictive control (MPC for short) is a control technique which, in closed-loop,
computes the control inputs by means of an optimization algorithm, which uses a
\textbf{model} of the system and measurements to \textbf{predict} future states and act
accordingly by choosing the "best" control action. 

\begin{equation}
\begin{aligned}
    \min_{u_0,...,u_{N-1}}{J(x,u)} \\
    \text{ subject to }
        \quad x_{k+1} = f(x_k,u_k) \\
        \quad x_0 = x(t)
        \quad 
\end{aligned}
\end{equation}

We call $J(x)$ the \textbf{cost function} and $g(x)$ the \textbf{constraints}.
In particular, the sequence of control actions is generated such that the cost
function is minimized over the \textbf{prediction horizon} by solving a
constrained optimization problem that depends on the evolution of the model over
the horizon itself. Then, the controller applies just the first action: in this
way the system has advanced one step, a new optimization problem with a new
initial state is produced, and the process goes on. The advantages of MPC are
many: it is a multivaribale controller, so it can control outputs by handling
simultaneously all the interactions between system variables; it can handle
constraints, so it allows to avoid possible undesired states; it predicts the
future states, allowing to incorporate their information in the actual control.
It is particulary useful for a real-time control that adapts to \textbf{changes
in the environment}.\\
When dealing with nonlinear dynamics, a nonlinear MPC (or NMPC) can be used to
capture more accurately the nonlinear behavior of a system. This entails a more
robust manipulation of both nonlinearities and uncertainties in dynamics models
and constraints. However, one of its highest problem is the computational power
it can require, especially when dealing with highly nonlinear systems. This
entails the need of a high-performance hardware to solve the optimization
problem in an acceptable time, but sometimes this is not enough. Especially in a
real environment, where the timeliness is fundamental to take a decision, fast
computations are of utmost importance.\\
The purpose of \cite{paper} is to develop a novel approach for a real-time NMPC
for an autonomous vehicle, which was tested in a race environment, where the
objective was to complete a lap in the minimum possible time, while also
enhancing computational efficiency. The architecture is a cascaded model
composed by a detailed model of the car for the near term, and a less complex
model used for a long horizon planning. This concept will be addressed more in
detail.\\
In this report we tried to replicate the architecture of the paper, and we
tested it in a simulation environment we created. Our purpose was to show the
mean computation time for the cascaded model is lower than the mean computation
time for a complete detailed model, with respect to the same length of the
horizon.

\subsection*{Report outline}

\section*{Related Work}


\section*{Methodology}

\begin{itemize}
    \item Overview/Introduction
    \begin{itemize}
        \item Concept of MPC for vehicle control
        \item Concept of serially cascaded models 
    \end{itemize}

    \item Paper
    \begin{itemize}
        \item Vehicle dynamic models
        \item NLP
    \end{itemize}    
\end{itemize}

\section*{Implementation}

\begin{itemize}
    \item Tools and libraries
    \item Description of implementation process
    \item Modifications or adaptations wrt the paper
\end{itemize}

\section*{Experimental Setup}

\begin{itemize}
    \item Simulation setting (track etc.)
    \item Different configuration scenarios
\end{itemize}

\section*{Results}

\begin{itemize}
    \item Guess what
\end{itemize}

\section*{Conclusion}

\begin{itemize}
    \item Take-away message
    \item Pitfalls and future work
\end{itemize}


\bibliographystyle{apalike}
\bibliography{references}
\end{document}
