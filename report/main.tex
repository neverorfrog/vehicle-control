\documentclass[a4paper, twocolumn, 11pt, twoside]{article}
\usepackage[english]{babel} %language
\usepackage[utf8]{inputenc} %input encoding
\usepackage{float} %position of floating objects
\usepackage{bookmark} %hyperlinks in pdf
\usepackage{subcaption}
\usepackage[T1]{fontenc}
\usepackage{lipsum} %placeholder text
\usepackage{amsthm} %theorems
% \usepackage{amsmath}
% \usepackage{mathtools}
% \usepackage{physics}
% \usepackage{xcolor}
% \usepackage{graphicx}
% \usepackage[left=23mm,right=13mm,top=35mm,columnsep=15pt]{geometry} 
% \usepackage{adjustbox}
% \usepackage{placeins}
% \usepackage{csquotes}
% \usepackage{algorithm} 
% \usepackage{algpseudocode}
% \usepackage{listings}
% \usepackage{graphicx}
% \usepackage{tikz}
% \usepackage[normalem]{ulem}
% \useunder{\uline}{\ul}{}

\title{Long-Horizon Vehicle Motion Planning and Control Through Serially Cascaded Model Complexity}
\author{}
\date{\today}

\begin{document}

\maketitle

\begin{abstract}
    We propose the implementation and experimentation of a motion planning and
    control framework for autonomous vehicles based on nonlinear
    model-predictive control.The work is mainly based on \cite{paper}. The code is available publicly at
    \href{https://github.com/neverorfrog/vehicle-control}{this GitHub
    repository}. 
\end{abstract}


\section{Introduction}

\begin{itemize}
    \item Overview of the paper
    \item Problem statement
    \item Literature review (?)
    \item Report outline
\end{itemize}

% Using the star (*) form of the sectioning command suppresses the numbering for
% that particular section or subsection.
\subsection*{Overview of the paper}


Nonlinear model predictive control (NMPC) is a powerful tool for the control of systems with a nonlinear dynamics 
that tries to predict the evolution of the system over a prediction window. It is based on a cost function, which must
be minimized at each time step according to a set of constraints.
One of its highest problem is the computational power it can require. This entails the need of a high-performance hardware
to solve the optimization problem in an acceptable time, but sometimes for high nonlinear dynamics systems this is not enough.
Especially in a real environment, where the timeliness is fundamental to take a decision.

\section{Methodology}

\begin{itemize}
    \item Concept of MPC for vehicle control
    \item Concept of serially cascaded models
\end{itemize}

\section{Implementation}

\begin{itemize}
    \item Tools and libraries
    \item Description of implementation process
    \item Modifications or adaptations wrt the paper
\end{itemize}

\section{Experimental Setup}

\begin{itemize}
    \item Simulation setting (track etc.)
    \item Different configuration scenarios
\end{itemize}

\section{Results}

\begin{itemize}
    \item Guess what
\end{itemize}

\section{Conclusion}

\begin{itemize}
    \item Take-away message
    \item Pitfalls and future work
\end{itemize}

\bibliographystyle{apalike}
\bibliography{references}
\end{document}
