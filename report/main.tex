\documentclass[a4paper, twocolumn, 11pt, twoside]{article}
\usepackage[english]{babel} %language
\usepackage[utf8]{inputenc} %input encoding
\usepackage{float} %position of floating objects
\usepackage{bookmark} %hyperlinks in pdf
\usepackage{subcaption}
\usepackage[T1]{fontenc}
\usepackage{lipsum} %placeholder text
\usepackage{amsthm} %theorems
% \usepackage{amsmath}
% \usepackage{mathtools}
% \usepackage{physics}
% \usepackage{xcolor}
% \usepackage{graphicx}
% \usepackage[left=23mm,right=13mm,top=35mm,columnsep=15pt]{geometry} 
% \usepackage{adjustbox}
% \usepackage{placeins}
% \usepackage{csquotes}
% \usepackage{algorithm} 
% \usepackage{algpseudocode}
% \usepackage{listings}
% \usepackage{graphicx}
% \usepackage{tikz}
% \usepackage[normalem]{ulem}
% \useunder{\uline}{\ul}{}

\title{Long-Horizon Vehicle Motion Planning and Control Through Serially Cascaded Model Complexity}
\author{}
\date{\today}

\begin{document}

\maketitle

\begin{abstract}
    We propose the implementation and experimentation of a motion planning and
    control framework for autonomous vehicles based on nonlinear
    model-predictive control.The work is mainly based on \cite{paper}. The code is available publicly at
    \href{https://github.com/neverorfrog/vehicle-control}{this GitHub
    repository}. 
\end{abstract}


\section{Introduction}

\begin{itemize}
    \item Overview of the paper
    \item Problem statement
    \item Literature review (?)
    \item Report outline
\end{itemize}

% Using the star (*) form of the sectioning command suppresses the numbering for
% that particular section or subsection.
\subsection*{Overview of the paper}


Nonlinear model predictive control (NMPC) is a powerful tool for the control of systems with a nonlinear dynamics 
that tries to predict the evolution of the system over a prediction window. It is based on a cost function, which must
be minimized at each time step according to a set of constraints.
One of its highest problem is the computational power it can require. This entails the need of a high-performance hardware
to solve the optimization problem in an acceptable time, but sometimes for high nonlinear dynamics systems this is not enough.
Especially in a real environment, where the timeliness is fundamental to take a decision.\\
The purpose of the article is to develop a novel approach for a real-time NMPC, which was tested in a race environment, where the 
objective was to complete a lap in the minimum possible time. The architecture is a cascaded model composed by a detailed model
of the car for the near term, and a less complex model used for a long horizon planning.

In this report we tried to replicate the architecture of the paper, and we tested it in a simulation environment we created.
Our purpose was to show the mean computation time for the cascaded model is lower than the mean computation time for a complete
detailed model, with respect to the same length of the horizon.

\section{Methodology}

\begin{itemize}
    \item Overview/Introduction
    \begin{itemize}
        \item Concept of MPC for vehicle control
        \item Concept of serially cascaded models 
    \end{itemize}

    \item Paper
    \begin{itemize}
        \item Vehicle dynamic models
        \item NLP
    \end{itemize}    
\end{itemize}

\subsection*{Overview}
MPC is a predictive control algorthm, which essentially predicts and plans for future states by continuously generating a series 
of control actions that will get the vehicle closer to its desired trajectory. In particular, the sequence of control actions 
is generated such that it minimizes the cost function over the established horizon by solving a constrained optimization problem
that depends on the current state. Then the controller applies just the first action: in this way the system has advanced one step,
a new optimization problem with a new initial state is produced, and the process goes on.
The advantages of MPC are many: it is a multivaribale controller, so it can control outputs by handling simultaneously all the 
interactions between system variables; it can handle constraints, so it allows to avoid possible undesired states;
it predicts the future states, allowing to incorporate their information in the actual control. \\
When dealing with nonlinear dynamics, a nonlinear MPC (or NMPC) can be used to capture more accurately the nonlinear behavior
of a system. This entails a more robust manipulation of both nonlinearities and uncertainties in dynamics models and constraints.

\section{Implementation}

\begin{itemize}
    \item Tools and libraries
    \item Description of implementation process
    \item Modifications or adaptations wrt the paper
\end{itemize}

\section{Experimental Setup}

\begin{itemize}
    \item Simulation setting (track etc.)
    \item Different configuration scenarios
\end{itemize}

\section{Results}

\begin{itemize}
    \item Guess what
\end{itemize}

\section{Conclusion}

\begin{itemize}
    \item Take-away message
    \item Pitfalls and future work
\end{itemize}

\bibliographystyle{apalike}
\bibliography{references}
\end{document}
